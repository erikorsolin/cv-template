\documentclass[10pt, letterpaper]{article}

% Packages:
\usepackage[
    ignoreheadfoot, % set margins without considering header and footer
    top=2 cm, % seperation between body and page edge from the top
    bottom=2 cm, % seperation between body and page edge from the bottom
    left=2 cm, % seperation between body and page edge from the left
    right=2 cm, % seperation between body and page edge from the right
    footskip=1.0 cm, % seperation between body and footer
    % showframe % for debugging 
]{geometry} % for adjusting page geometry
\usepackage{titlesec} % for customizing section titles
\usepackage{tabularx} % for making tables with fixed width columns
\usepackage{array} % tabularx requires this
\usepackage[dvipsnames]{xcolor} % for coloring text
\definecolor{primaryColor}{RGB}{0, 79, 144} % define primary color
\usepackage{enumitem} % for customizing lists
\usepackage{fontawesome5} % for using icons
\usepackage{amsmath} % for math
\usepackage[
    pdftitle={Erik Orsolin CV},
    pdfauthor={Erik Orsolin},
    pdfcreator={LaTeX with RenderCV},
    colorlinks=true,
    urlcolor=primaryColor
]{hyperref} % for links, metadata and bookmarks
\usepackage[pscoord]{eso-pic} % for floating text on the page
\usepackage{calc} % for calculating lengths
\usepackage{bookmark} % for bookmarks
\usepackage{lastpage} % for getting the total number of pages
\usepackage{changepage} % for one column entries (adjustwidth environment)
\usepackage{paracol} % for two and three column entries
\usepackage{ifthen} % for conditional statements
\usepackage{needspace} % for avoiding page brake right after the section title
\usepackage{iftex} % check if engine is pdflatex, xetex or luatex


% Ensure that generate pdf is machine readable/ATS parsable:
\ifPDFTeX
    \input{glyphtounicode}
    \pdfgentounicode=1
    % \usepackage[T1]{fontenc} % this breaks sb2nov
    \usepackage[utf8]{inputenc}
    \usepackage{lmodern}
\fi



% Some settings:
\AtBeginEnvironment{adjustwidth}{\partopsep0pt} % remove space before adjustwidth environment
\pagestyle{empty} % no header or footer
\setcounter{secnumdepth}{0} % no section numbering
\setlength{\parindent}{0pt} % no indentation
\setlength{\topskip}{0pt} % no top skip
\setlength{\columnsep}{0cm} % set column seperation
\makeatletter
\let\ps@customFooterStyle\ps@plain % Copy the plain style to customFooterStyle
\patchcmd{\ps@customFooterStyle}{\thepage}{
    \color{gray}\textit{\small Erik Orsolin de Paula - Page \thepage{} of \pageref*{LastPage}}
}{}{} % replace number by desired string
\makeatother
\pagestyle{customFooterStyle}

\titleformat{\section}{\needspace{4\baselineskip}\bfseries\large}{}{0pt}{}[\vspace{1pt}\titlerule]

\titlespacing{\section}{
    % left space:
    -1pt
}{
    % top space:
    0.3 cm
}{
    % bottom space:
    0.2 cm
} % section title spacing

\renewcommand\labelitemi{$\circ$} % custom bullet points
\newenvironment{highlights}{
    \begin{itemize}[
        topsep=0.10 cm,
        parsep=0.10 cm,
        partopsep=0pt,
        itemsep=0pt,
        leftmargin=0.4 cm + 10pt
    ]
}{
    \end{itemize}
} % new environment for highlights

\newenvironment{highlightsforbulletentries}{
    \begin{itemize}[
        topsep=0.10 cm,
        parsep=0.10 cm,
        partopsep=0pt,
        itemsep=0pt,
        leftmargin=10pt
    ]
}{
    \end{itemize}
} % new environment for highlights for bullet entries


\newenvironment{onecolentry}{
    \begin{adjustwidth}{
        0.2 cm + 0.00001 cm
    }{
        0.2 cm + 0.00001 cm
    }
}{
    \end{adjustwidth}
} % new environment for one column entries

\newenvironment{twocolentry}[2][]{
    \onecolentry
    \def\secondColumn{#2}
    \setcolumnwidth{\fill, 4.5 cm}
    \begin{paracol}{2}
}{
    \switchcolumn \raggedleft \secondColumn
    \end{paracol}
    \endonecolentry
} % new environment for two column entries

\newenvironment{header}{
    \setlength{\topsep}{0pt}\par\kern\topsep\centering\linespread{1.5}
}{
    \par\kern\topsep
} % new environment for the header

\newcommand{\placelastupdatedtext}{% \placetextbox{<horizontal pos>}{<vertical pos>}{<stuff>}
  \AddToShipoutPictureFG*{% Add <stuff> to current page foreground
    \put(
        \LenToUnit{\paperwidth-2 cm-0.2 cm+0.05cm},
        \LenToUnit{\paperheight-1.0 cm}
    ){\vtop{{\null}\makebox[0pt][c]{
        \small\color{gray}\textit{Last updated in August 2025}\hspace{\widthof{Last updated in September 2024}}
    }}}%
  }%
}%

% save the original href command in a new command:
\let\hrefWithoutArrow\href

% new command for external links:
\renewcommand{\href}[2]{\hrefWithoutArrow{#1}{\ifthenelse{\equal{#2}{}}{ }{#2 }\raisebox{.15ex}{\footnotesize \faExternalLink*}}}


\begin{document}
    \newcommand{\AND}{\unskip
        \cleaders\copy\ANDbox\hskip\wd\ANDbox
        \ignorespaces
    }
    \newsavebox\ANDbox
    \sbox\ANDbox{}

    \placelastupdatedtext
    \begin{header}
        \textbf{\fontsize{24 pt}{24 pt}\selectfont Erik Orsolin de Paula}

        \vspace{0.3 cm}

        \normalsize
        \mbox{{\color{black}\footnotesize\faMapMarker*}\hspace*{0.13cm}Florianópolis}%
        \kern 0.25 cm%
        \AND%
        \kern 0.25 cm%
        \mbox{\hrefWithoutArrow{mailto:depaulaerik670@gmail.com}{\color{black}{\footnotesize\faEnvelope[regular]}\hspace*{0.13cm}depaulaerik670@gmail.com}}%
        \kern 0.25 cm%
        \AND%
        \kern 0.25 cm%
        \mbox{\hrefWithoutArrow{https://linkedin.com/in/erik-orsolin}{\color{black}{\footnotesize\faLinkedinIn}\hspace*{0.13cm}Erik Orsolin}}%
        \kern 0.25 cm%
        %\AND%
        %\kern 0.25 cm%
        %\mbox{\hrefWithoutArrow{https://github.com/erikorsolin}{\color{black}{\footnotesize\faGithub}\hspace*{0.13cm}erikorsolin}}%
    \end{header}

    \vspace{0.3 cm - 0.3 cm}


    \section{Resumo}



        
        \begin{onecolentry}
            Estudante de Ciência da Computação com sólida experiência em testes e desenvolvimento de software. Apaixonado por tecnologia, possuo grande interesse em design de sistemas e arquitetura de software, com foco em soluções escaláveis e bem estruturadas.
        \end{onecolentry}

        \vspace{0.2 cm}


    \section{Educação}



        
        \begin{twocolentry}{
            
            
        \textit{Mar 2022 – Dez 2026}}
            \textbf{Universidade Federal de Santa Catarina}

            \textit{Bach. em Ciência da Computação}
        \end{twocolentry}

        \vspace{0.10 cm}
        \begin{onecolentry}
            \begin{highlights}
                \item \textbf{Disciplinas:} Sistemas Operacionais, Estruturas de Dados, Algoritmos e Otimização, Teoria dos Grafos, Redes de Computadores, entre outras.
            \end{highlights}
        \end{onecolentry}



    
    \section{Experiencia}



        
        \begin{twocolentry}{
        \textit{Florianópolis, SC}    
            
        \textit{Ago 2025 – Presente}}
            \textbf{Desenvolvedor de Software}
            
            \textit{SRS Desenvolvimento de Sistemas}
        \end{twocolentry}

        \vspace{0.10 cm}
        \begin{onecolentry}
            \begin{highlights}
                \item Desenvolvi e mantive funcionalidades do ERP da maior distribuidora de autopeças do sul do Brasil, utilizando Delphi e Firebird.
                \item Refatorei partes críticas do sistema em Delphi, migrando classes antigas para padrões modernos de design como utilização de ORMs.
                \item Desenvolvi telas para novas funcionalidades da aplicação Desktop do ERP utilizando a biblitoeca VCL.
                \item Corrigi bugs em APIs internas do ERP, melhorando a disponibilidade do sistema como um todo.
                
            \end{highlights}
        \end{onecolentry}


        \vspace{0.2 cm}

        \begin{twocolentry}{
        \textit{Florianópolis, SC}    
            
        \textit{Mar 2025 – Ago 2025}}
            \textbf{Quality Assurance}
            
            \textit{SRS Desenvolvimento de Sistemas}
        \end{twocolentry}

        \vspace{0.10 cm}
        \begin{onecolentry}
            \begin{highlights}
                \item Atuei diretamente no desenvolvimento e garantia de qualidade de um chatbot de atendimento ao cliente utilizando o RASA como motor de NLP.
                \item Desenvolvi e mantive suítes de testes automatizados de integração em Python (Pytest), cobrindo \text{80\%} das funcionalidades do chatbot e diminuindo em \text{65\%} a quantidade de bugs em produção. 
                \item Implementei logs detalhados nos testes automatizados para aumentar a observabilidade do chatbot, facilitando a detecção de falhas e acelerando o processo de debugging.
                \item Integrei os testes automatizados ao pipeline de CI/CD do GitLab, configurando regras para sua execução em merge requests.
            \end{highlights}
        \end{onecolentry}


        \vspace{0.2 cm}

        \begin{twocolentry}{
        \textit{Remoto}    
            
        \textit{Jun 2024 – Mar 2025}}
            \textbf{Quality Assurance}
            
            \textit{Laboratório Bridge}
        \end{twocolentry}

        \vspace{0.10 cm}
        \begin{onecolentry}
            \begin{highlights}
                \item Atuei na garantia de qualidade de três aplicativos Android críticos do Ministério da Saúde -- e-SUS Território, e-SUS Vacinação e e-SUS Atividade Coletiva -- que juntos totalizam mais de 100 mil downloads e são utilizados em larga escala por profissionais da saúde.
                \item Desenvolvi testes automatizados end-to-end (E2E) com Espresso em Kotlin, simulando interações reais de usuários para validar o comportamento completo dos aplicativos Android.
                \item Participei ativamente de cerimônias Scrum, colaborando com a equipe de desenvolvimento na identificação e resolução de bugs e no alinhamento de requisitos técnicos.
                \item Utilizei Git/GitHub Projects para reportar e gerenciar o ciclo de vida dos bugs, garantindo a rastreabilidade e a comunicação eficiente com os desenvolvedores.
            \end{highlights}
        \end{onecolentry}


                \vspace{0.2 cm}

        \begin{twocolentry}{
        \textit{Florianópolis, SC}    
            
        \textit{Ago 2022 – Mai 2024}}
            \textbf{Bolsista de Iniciação Científica}
            
            \textit{Programa de Educação Tutorial}
        \end{twocolentry}

        \vspace{0.10 cm}
        \begin{onecolentry}
            \begin{highlights}
                \item Desenvolvi um script em Python para automatizar a geração e o envio de certificados de eventos, otimizando um processo manual e reduzindo o tempo de trabalho da equipe.
                \item Colaborei em projetos de pesquisa, apliquei lógica de programação e elaborei
                relatórios técnicos, desenvolvendo forte capacidade de análise e escrita.
                \item Ministrei mentorias e minicursos sobre tecnologia para outros estudantes, fortalecendo habilidades de comunicação e ensino técnico.
                \item Elaborei atividades de inglês para outros estudantes da UFSC.
                
            \end{highlights}
        \end{onecolentry}

    
    \section{Tecnologias}



        
        \begin{onecolentry}
            \textbf{Linguagens:} Delphi 12, Python, C\#, C++, SQL
        \end{onecolentry}

        \vspace{0.2 cm}

        \begin{onecolentry}
            \textbf{Tecnologias gerais:} .NET, RAD Studio, PostgreSQL, Firebird, Git, Make, Docker
        \end{onecolentry}


    

\end{document}